
\section{Implementation}
In the beginning and during the project the group were given ideas and guidelines on which practices that could be applicable to a software development project. Depending on the type and size of a project some practices suit better than others and some practices needs adjustments to gain value for the group. In this section the choices, adjustments and implementation of project practices will be presented.

As suggested in the early lectures of the course the group has mainly used the SCRUM method during the project. Because the project is relatively short, stretching just over 4 iterations at 2 weeks each it was decided that this was a suitable agile method. As preparation to use SCRUM the group members plowed through the SCRUM guide written by Kniberg \cite{kniberg_scrum_2015}. According to the SCRUM guide it's valuable to have a SCRUM master in the group, directing the work in a SCRUM manner but since none of the group members were familiar with the method this was disregarded and the group made an effort to interpret the workflow in a democratic and suitable way for the project. This led to some modifications of the method, the first addressed hear is the physical SCRUM work board. 

The group valued the flexibility of having a cloud based SCRUM board and therefore opted to use the GitHub KanBan board, modified to suit the SCRUM method. After the first iteration the group also added a complementary physical work board  because that facilitated the SCRUM meetings which was done in person at the workstation in Kista. 

The SCRUM method suggests to organise daily meetings which was initially adopted by the group but later revised to just keep meetings on days when in person meetings were planned. Because the group didn't work full time with the project there were not enough work done each day to support daily meetings. 

Other than the work board, GitHub provided the group with several tools during the project. The group used a GitHub page to gather information and links to associated documents for the project. Repositories ("repo") were used to divide the project in different modules, e.g. one repo for the IOT device and one repo for the webpage. Because of the backgrounds of the groupmembers, coming from different programs at KTH the group also chose to work in two teams. Student from the more hardware focused program TIEDB worked in a hardware team while the students from TIDAB and TIDEA joined the software team. The teams could then work independently in their corresponding repositories. 

It is good practice to use a Version Control System (VCS) for software projects and the group naturally chose to use GitHub for this \cite{ian_sommerville_software_nodate}. GitHub uses the distributed VCS GIT which can be controlled via command line but the group members chose to use the "GitHub Desktop" desktop application which has a friendlier interface. 

Next follows some of the implementations the group members have been doing in the area of their roles. 

\subsection{Project Manager (Caroline)}
The project manager has the responsibility to lead and coordinate work in the project group. In this role Caroline has been hosting the weekly and daily SCRUM meetings with the group as well as taking greater responsibility during the demonstrations of the product. Caroline has also taken a major role in the writing of the project definition document. 

\subsection{Stakeholder representation (Natasha)}
In a normal project this is the product owner who attends to the group meetings to align the requirements with the work being done by the group. In this school project there are no real stakeholders and therefore the group have been working with a MuSCoW requirements model which the group created \cite{eklund_arbeta_2010}. Natasha has been responsible for making sure the work done by the group has been in compliance with the MuSCoW. She has also been responsible for developing the vision document. 

\subsection{Architecture (Jesper)}


\subsection{Development (Wilhelm)}


\subsection{Testing (Arif)}


\subsection{Sustainability (Gustav)}

