
\section{discussion}
The agile method Scrum has been applied to the research project; several nuanced procedures within the methodology have been tested and evaluated by the group members regarding their usefulness within a small IT project. Generally, procedures requiring much time were less beneficial for the group, where the invested time was relatively large compared to the actual benefit.[reference]. Additionally to Scrum, the project group was divided into two teams: the Web team and the IoT team. For our group, this method worked sufficiently and was a necessity since all group members did not have the same knowledge or skills. Still, the two teams managed to communicate well throughout the project. In retrospect, it would have been challenging to do this differently. Since we paired up in two teams, the members did not have to familiarize themselves with the other group's code or issues, and time could, in this way, be saved.

During the first two weeks, the project group chose to implement daily scrum meetings but quickly realized that having it every day was excessive for a halftime project, which led to changing the Scrum meetings to two times per week during the common-time meetings instead. However, the group could see the profit of having daily scrum meetings if it was a full-time project. It is a great forum to find issues quickly and prevent team members from working excessively long on unsuitable tasks. 

The physical and digital scrum boards were excellent tools to structure the workload. The group chose to do the digital scrum board on GitHub, which operated more promisingly than anticipated. All three members on the "Web team" had worked with GitHub before, while the "IoT team" had not. Therefore, the learning process was longer for the second team but worth it when everything worked smoothly later on in the project, thanks to the tool. The tasks on scrum boards were created during sprint planning meetings. These meetings were held in person in the working area at KTH. It enabled the group to create cohesion and team spirit. As there was uncertainty in the sprint planning methodology initially, the tasks did not turn out perfect at first. Throughout the course, the group developed procedures during the sprint planning meetings, which eventually resulted in well-defined tasks.

\subsection{Method discussion}
Validity
Reliability

\subsection{Result discussion}
Validity
Reliability

\subsection{Bidrag till vetenskaplighet, ingenjörserfarenhet (studenterfarenhet?)}
Framtida förbättringar\dots
