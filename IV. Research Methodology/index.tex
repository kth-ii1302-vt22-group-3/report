\section{Research Methods}


The research methods describe the methods that will answer the main research question, listed below and followed by a method description of the different project methods.


\subsection{Questions to be answered in the research}

\begin{enumerate}
    \item{How should one assess/report whether some project method or practicum is suitable for an IoT project?}
    \item{How to categorize, select and name project methods?}
    \item{What are the responsibility roles used in a project?}
    \item{What is a project built of, and which methods/practices for examining and asserting the knowledge?}
\end{enumerate}

\subscetion{Method Description}

The different project method and working method can affect the outcome of the development of a end product.
The determination of which method works best in the project groups depends on the experience one have in the the different project methods. 
The different project methods exists so that the development can focus on different scopes and all comes together at the end in the end-goal.

The selection of project methods were introduces very early in the course and the project layout was based on iterations.
The suitable project methods lead down to working with Scrum\cite{atlassianScrum} and how to utilize it for the future iterations. 
Each iteration had a spring goal which was set by the project group but also from the examinator of this course. 
The sprint planning was formed around what were the pre-defined requirements from the examinator for the iteration, and the addition of construction details from the project group.

The project method, Scrum, was used for the compatibility of examinator's requirements for this cource and the layout of this course.
The project group could utilized a simplier form like Kanban\cite{atlassianKanban}, which would lead to a more horizontal development where all task are in one side and moves to the other-side through the coloums.

Scrum is built on having sprint goals and defined task in the beginning of the iteration which then are focused on completing them before the iteration ends.
This mindset led to the realization that Scrum would fit this project better, as the project was under a limited time and on a single minimal-viable-product by the end of the course.
One iteration were composed of sprint planning, daily scrum meetings, sprint retrospective, Game-Score-Card.
\textbf{Sprint planning}, is about setting the current iterations goal and which stories to complete by the end of the iteration. 
To make it easier for the project members, the stories were broken down to smaller task with would be determine that it could be completed in a short amount of time compared to having to complete the whole story.
\textbf{Daily scrum meeting}, is about informing the other project member on how the individuals workload has been and for help each other move forward.
\textbf{Sprint retrospective}, is about looking back on the completed iteration and understand what went good, could be better, and what needs to be stopped.
\textbf{Game Score Card}, is an academic addition to the iteration, and was used for grading the itself and the project members contribution to the iteration.


