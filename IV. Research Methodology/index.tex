\section{Research Methods}


The research methods describe the methods that will answer the main research question, listed below and followed by a method description of the different project methods.


\subsection{Questions to be answered in the research}

\begin{enumerate}
    \item{How should one assess\/report whether some project method or practicum is suitable for an IoT project?}
    \item{How to categorize, select and name project methods?}
    \item{What are the responsibility roles used in a project?}
    \item{What is a project built of, and which methods/practices for examining and asserting the knowledge?}
\end{enumerate}

\subsection{Method Description}

In this section we describe the different project methods, and the methods that are used in the project.
The two common methods are the \textit{SCRUM} and the \textit{Kanban} method.
Both methods are based on the same principles, but the SCRUM method is more formal structure that focuses on the development of the project, while the Kanban method focuses on the implementation of the project.
The difference between the two methods is that the SCRUM method is based on the \textit{Sprint} concept, while the Kanban method is based on the \textit{Task} concept.

Scrum\cite{atlassianScrum} is build upon a structure of \textit{Sprints}, which are defined by the \textit{Sprint Goal}, \textit{Sprint Backlog}, \textit{Sprint Review} and \textit{Sprint Retrospective}.
Sprint goal, is about the project's main goal, while the Sprint Backlog is a list of the tasks that need to be done in the Sprint.
The workboard is build on states which the tasks are in, and the workboard is used to show the progress of the project and the board is generated for every iteration.
The workboard coloums states are:
\begin{enumerate}
    \item{To Do}
    \item{In Progress}
    \item{In Review}
    \item{Done}
\end{enumerate}

The Sprint Retrospective is a list of the tasks that were done in the Sprint, and the Sprint Review is a list of the tasks that were not done in the Sprint.
The retrospective is about understanding the why and how the tasks were completed.
During the retrospective you state what was good, what was bad, what could be improved and what could be done better, and implement the changes into the next Sprint.


The Kanban\cite{atlassianKanban} method is based on the \textit{Task} concept, which is defined by the \textit{Task Board}, \textit{Task Backlog}, \textit{Task Review} and \textit{Task Retrospective}.
The Task Board is a list of the tasks that need to be done in the project.
The Task Backlog is a list of the tasks that are in progress.
The Task Review is a list of the tasks that were done in the project and are waiting for review.
The Task Retrospective is a list of the tasks that were done in the project, and the Task Review is a list of the tasks that were not done in the project.

The concept of Kanban and Scrum is very similiar as described above.


As the course material and course requirements were optimized for utilizing the Scrum\cite{atlassianScrum} method, with all the iterations and their requirements for each iteration.
The course material is divided into the following sections:
\begin{enumerate}
    \item{Iteration 1, Introduction of course and project requirements.}
    \item{Iteration 2, Elaboration of the project requirements.}
    \item{Iteration 3, Construction of the project requirements.}
    \item{Iteration 4, Construction of the project requirements.}
    \item{Iteration 5, Transition of the project requirements.}
\end{enumerate}

The project group members was divided into the following groups:
\begin{enumerate}
    \item{Project Manager, was tasked to be responsible for the presentation of the product and write a documenation about the project definition.}
    \item{Customer and Requirements Manager, was tasked to be responsible for the defining the vision and business case with the product is going to be fulfill.}
    \item{Architecture, was tasked to come up with a architecture for the whole product.How all the components communicate with the others modules which then leads to the end product primary use case.}
    \item{Construction and development Manager, is responsible for writing a component description that are based on the architectual design.}
    \item{Test Manager, is responsible for creating test cases which full fufill the use case defined in the project.}
    \item{Sustainability\/Work Environment Manager, is responsible for creating a sustainability development description.}
\end{enumerate}


In this course the project group focus on the Scrum\cite{atlassianScrum} method for developing the product.
For structuring the project the project group implemented the workboard in GitHub Project Management tool\cite{github-project-board}.
The workboard was divided into different tabs which provided different information to the external viewers and the project members.
The workboard tabs was divided into the following tabs:
\begin{enumerate}
    \item{Project Planner List, used for an overview of all task in the project with detailed metadata.}
    \item{Project Planner Table, used for external presentation of the project plan and their sprint goals.}
    \item{Current Sprint List, used for an overview of all task in the current sprint with detailed metadata.}
    \item{Current Sprint Table, used for external presentation of the current sprint plan and their sprint progress.}
    \item{My To Do's, filtered tasks which have been assigned to a specific user.}
    \item{Stories Table, used for having an overview of all task connected to a specific story.}
\end{enumerate}

Every iteration was followed by a sprint planning meeting, where the group stated the sprint goals, and the sprint backlog for the iteration.
The sprint ended with a sprint demo of the current iteration progress, the sprint retrospective of the iteration progress, and a Game Score Card for grading the performance on itself and on the project members. 