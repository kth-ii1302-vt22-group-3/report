
\begin{abstract}
Syfte och mål med kursen - "Vad är en bra projektmetod för små IT-projekt?" Kursens metod för att uppnå
kursens syfte och mål. Resultat av kursens metod - uppfylls syfte och mål med kursen.\\
Kan undersökningsfrågan besvaras?

What is a good project method for small IT projects? 

To give students the knowledge and brief experience about project methods used in small IT projects ... Role play ... seminar etc ...

This method gives the students a good knowledge and a somewhat ____ experience in ... However other methods are not tested = students can only say if his or her group used the SCRUM method in a way that felt good for him or her, not if any other would have worked better, if _____ or if they even followed the SCRUM method as it is supposed/meant to. 


\end{abstract}

\begin{IEEEkeywords}
Group project, IoT, Project method, Scrum, KTH EECS. 
\end{IEEEkeywords}

\section{Om detta dokument och undersökning}
Vem är läsaren? Dokumentets disposition (istället för innehållsförteckning). Vilken trovärdighet har innehållet?

Testar att citera \cite{eklund_arbeta_2010}

Projektets resultat (framtagen produkt) framgår av bilagorna.

Bilagor.
