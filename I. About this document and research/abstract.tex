
\begin{abstract}
- Course at KTH ICT (Information and Communication Technology) 
Purpose and goals with the course - "What is a good project method for small IT projects?". To give students the knowledge and brief experience about project methods used in small IT projects ... Role play ... seminar etc ... Examiner and other teachers following up on the group to see that they are putting in the hours expected and giving feedback on the demo at the end of each iteration. (No quality check of the documentation or progress with the product is done until the course is finished, unless the group asks specific questions). 

This method gives the students a good knowledge and a somewhat ____ experience in ... However other methods are not tested, which means students can only say if his or her group used the SCRUM method in a way that felt good for him or her, not if any other method would have worked better, if _____ or if they even followed the SCRUM method as it is supposed/meant to. 

With this in mind the investigation consisting (?) of this project can answer the question "Is Scrum a good project method for this project group during this course project?" but not if the group in fact implemented Scrum as it is intended or if Scrum would be better than any other project method. 




Syfte och mål med kursen - "Vad är en bra projektmetod för små IT-projekt?" Kursens metod för att uppnå
kursens syfte och mål. Resultat av kursens metod - uppfylls syfte och mål med kursen.\\
Kan undersökningsfrågan besvaras?






\end{abstract}

\begin{IEEEkeywords}
Group project, IoT, Project method, Scrum, KTH EECS. 
\end{IEEEkeywords}

\section{About this document and investigation}
Vem är läsaren? Dokumentets disposition (istället för innehållsförteckning). Vilken trovärdighet har innehållet?

Testar att citera \cite{noauthor_hallbar_nodate}

Projektets resultat (framtagen produkt) framgår av bilagorna.

Bilagor.
